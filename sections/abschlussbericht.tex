\subsection{Zusammenfassung der Projektziele}
Das Projektziel war die Entwicklung eines selbstfahrenden Miniatur-Golfcars, das in der Lage ist, Golfbälle autonom zu erkennen, aufzunehmen und an einem definierten Ort abzulegen. Die Website diente als Plattform zur Präsentation des Projekts, zur Bereitstellung von Informationen und zur Interaktion mit dem Fahrzeug über eine Live-Steuerung.

\subsection{Erfolge und Herausforderungen}
\subsubsection{Technische Herausforderungen und Lösungen}
Die Konstruktion des Fahrzeugs und die Entwicklung der Website brachten verschiedene technische Herausforderungen mit sich:
\begin{itemize}
    \item \textbf{Platzierung der Ultraschallsensoren:} Die optimale Platzierung der Sensoren erwies sich als schwierig, wurde jedoch durch wiederholte Tests und Anpassungen des Chassis erfolgreich gelöst.
    \item \textbf{Verwendung von Autodesk Inventor:} Trotz der Komplexität des Programms ermöglichte die Verwendung von professioneller CAD-Software eine präzise und modulare Konstruktion, die entscheidend für den Erfolg des Projekts war.
    \item \textbf{Sicherheitsvorfall:} Ein Brute-Force-Angriff führte zu einer Überarbeitung der Sicherheitsmaßnahmen. Durch die Implementierung von Fail2Ban und die Aktualisierung der Sicherheitseinstellungen konnten weitere Vorfälle verhindert werden.
\end{itemize}

\subsubsection{Erfolge}
\begin{itemize}
    \item \textbf{Design und Konstruktion:} Das Fahrzeug wurde erfolgreich im Stil des Cybertrucks entworfen und im Maßstab 1:17 umgesetzt. Der 3D-Druck außerhalb der Schule ermöglichte eine hochwertige und genaue Herstellung der Teile.
    \item \textbf{Website und Interaktivität:} Die Website hat sich als robuste Plattform für die Projektpräsentation und Interaktion erwiesen. Insbesondere die Integration von Google Analytics und SEO-Optimierung verbesserten die Sichtbarkeit und Nutzbarkeit der Website.
\end{itemize}

\subsection{Was wir gelernt haben}
Dieses Projekt bot wertvolle Lerneinblicke in mehreren Bereichen:
\begin{itemize}
    \item \textbf{Teamarbeit und Projektmanagement:} Die Anwendung von Kanban mittels Trello verbesserte unsere Fähigkeit, Aufgaben zu organisieren und den Projektfortschritt transparent zu gestalten.
    \item \textbf{Technische Fähigkeiten:} Der Umgang mit komplexer CAD-Software, Programmierung und Netzwerkmanagement sind Fähigkeiten, die wir während des Projekts erheblich verbessern konnten. Die Anwendung dieser Fähigkeiten wurde durch frühere Kurse in Telekommunikation und Schaltkreisdesign, die wir in der dritten und vierten Klasse absolviert hatten, unterstützt.
    \item \textbf{Problembehandlung und Kreativität:} Die Lösung unerwarteter technischer Probleme erforderte ein kreatives Denken und hat unsere Problemlösungskompetenzen gestärkt.
    \item \textbf{Anwendung theoretischen Wissens:} Die praktische Anwendung von theoretischem Wissen aus früheren Kursen in Telekommunikation und Hardware-Engineering half uns, die Herausforderungen bei der Entwicklung von Schaltkreisen und der Integration von Hardwarekomponenten effektiv zu bewältigen. Dies zeigte sich besonders in der Fähigkeit, verschiedene Sensorsysteme und Kommunikationsprotokolle zu integrieren und zu optimieren.
\end{itemize}

\subsection{Ausblick und zukünftige Schritte}
Die Erfahrungen und Ergebnisse dieses Projekts bilden eine solide Grundlage für zukünftige technische Unternehmungen. Die Veröffentlichung des Quellcodes auf GitHub wird nicht nur die Transparenz und Zugänglichkeit des Projekts fördern, sondern auch andere Bildungseinrichtungen und Technikbegeisterte zur Nachahmung und Weiterentwicklung anregen.

\subsection{Schlusswort}
Wir danken allen Beteiligten, die zum Erfolg dieses Projekts beigetragen haben. Es war eine bereichernde Erfahrung, die zeigt, wie technische Bildung praktisch angewendet werden kann, um innovative Lösungen zu realisieren. Wir sind gespannt auf die zukünftigen Entwicklungen und die weiterführende Nutzung unserer Ergebnisse.

