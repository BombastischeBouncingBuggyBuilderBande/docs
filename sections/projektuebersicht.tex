Die Entwicklung des selbstfahrenden Golfcars ist ein ambitioniertes Projekt, das von 6B Engineering UG initiiert wurde, um innovative Lösungen im Bereich der autonomen Fahrzeuge zu erforschen und zu demonstrieren. Dieses Projekt kombiniert fortschrittliche Technologien aus den Bereichen Robotik, maschinelles Lernen und mechanisches Design, um ein Fahrzeug zu entwickeln, das fähig ist, Golfbälle selbstständig zu lokalisieren, aufzusammeln und zu transportieren.

\subsection{Projektziele}
Die Hauptziele des Projekts umfassen:
\begin{itemize}
  \item Entwicklung eines voll funktionsfähigen Prototyps eines selbstfahrenden Golfcars, das in der Lage ist, autonom auf einem Golfplatz zu operieren.
  \item Integration von Sensortechnologien zur Hinderniserkennung und -navigation.
  \item Erstellung einer benutzerfreundlichen Schnittstelle zur Überwachung und Steuerung des Fahrzeugs über eine dedizierte Website.
  \item Förderung der technischen Bildung und Inspiration für zukünftige Ingenieure durch offenen Zugang zu Projektressourcen und Dokumentationen.
\end{itemize}

\newpage

\subsection{Projektkomponenten}
Das Projekt gliedert sich in mehrere Schlüsselkomponenten:
\begin{itemize}
  \item \textbf{Fahrzeugdesign und -konstruktion:} Entwurf und 3D-Druck eines maßstabsgetreuen Modells des Cybertrucks, angepasst an die technischen Komponenten wie Raspberry Pi, Sensoren und Antriebseinheiten.
  \item \textbf{Softwareentwicklung:} Programmierung der Steuerungslogik, einschließlich Algorithmen für autonomes Fahren und maschinelles Lernen zur Objekterkennung.
  \item \textbf{Website-Entwicklung und -Hosting:} Aufbau einer Plattform zur Interaktion mit dem Fahrzeug und zur Bereitstellung von Projektinformationen und Updates.
\end{itemize}

\subsection{Erwartete Ergebnisse}
Das Projekt ist darauf ausgerichtet, am 09. Mai 2024 im Rahmen einer Schulveranstaltung präsentiert zu werden. Die erwarteten Ergebnisse umfassen:

\begin{itemize}
  \item \textbf{Demonstration des Prototyps:} Vorstellung des voll funktionsfähigen Prototyps des selbstfahrenden Golfcars, das die Fähigkeit demonstriert, autonom Golfbälle zu lokalisieren, aufzusammeln und zu transportieren.
  \item \textbf{Dokumentation und Lernressourcen:} Bereitstellung einer umfassenden Dokumentation des Projektverlaufs, der verwendeten Technologien und der Entwicklungsschritte. Diese Dokumentation dient als Lernressource für andere Schüler und Lehrkräfte.
  \item \textbf{Feedback und Weiterentwicklung:} Sammeln von Feedback von Lehrern, Schülern und externen Gästen während der Präsentation, um Verbesserungsvorschläge für zukünftige Iterationen des Projekts zu erhalten.
\end{itemize}